%!TEX root = ../template.tex
%%%%%%%%%%%%%%%%%%%%%%%%%%%%%%%%%%%%%%%%%%%%%%%%%%%%%%%%%%%%%%%%%%%%
%% abstract-en.tex
%% NOVA thesis document file
%%
%% Abstract in English([^%]*)
%%%%%%%%%%%%%%%%%%%%%%%%%%%%%%%%%%%%%%%%%%%%%%%%%%%%%%%%%%%%%%%%%%%%

\typeout{NT FILE abstract-en.tex}

Software development has had a significant impact on the world, but sometimes it has not 
been for the better. There are many examples of software failures that have caused terrible 
outcomes. To mitigate this, software testers have developed many techniques that require 
significant human effort and time. Although these efforts have been successful in many cases, 
there are still many software failures.

The work we propose in this dissertation aims to improve the quality of software by resorting 
to formal verification techniques, not as a replacement for testing, but as an improved tool to 
help build more reliable software. Our work is based on the use of \monitors, a technique that 
uses formal verification to check the correctness of software. This is employed by the use 
of static verification and, for the parts of the code that were unable to be checked, \rac 
or \textit{Runtime Assertion Checking}. Tools like \gospellang, \ortac, and \cameleer are 
used to implement this technique. \why is also essential to provide the necessary support 
for this work.

Here, we provide some background for \ocaml - the language we will focus on, formal verification 
itself, \gospellang, \ortac, \cameleer, and \why. We also present the work we have done in a 
\texttt{Queue} example, showing promising results in the implementation of the \monitors. 

Finally, we present some conclusions and future work.

\keywords{
  \ocaml \and
  Formal Verification \and
  Dynamic Verification \and
  \gospellang \and
  \ortac \and
  \cameleer \and
  \why
}
