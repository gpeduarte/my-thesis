%!TEX root = ../template.tex
%%%%%%%%%%%%%%%%%%%%%%%%%%%%%%%%%%%%%%%%%%%%%%%%%%%%%%%%%%%%%%%%%%%%
%% abstract-pt.tex
%% NOVA thesis document file
%%
%% Abstract in Portuguese
%%%%%%%%%%%%%%%%%%%%%%%%%%%%%%%%%%%%%%%%%%%%%%%%%%%%%%%%%%%%%%%%%%%%

\typeout{NT FILE abstract-pt.tex}

O desenvolvimento de software tem tido um impacto significativo no mundo, mas, por vezes, esse 
impacto não foi positivo. Há muitos exemplos de falhas de software que causaram consequências 
desastrosas. Para mitigar esse problema, \textit{testers} de software desenvolveram várias técnicas 
que exigem um esforço humano considerável e tempo. Embora esses esforços tenham sido 
bem-sucedidos em muitos casos, ainda ocorrem muitas falhas de software.

O trabalho que propomos nesta dissertação visa melhorar a qualidade do software recorrendo 
a técnicas de verificação formal, não como um substituto de testes, mas como uma ferramenta 
aprimorada para ajudar a construir software mais confiável. O nosso trabalho baseia-se no 
uso de \monitors, uma técnica que utiliza verificação formal para verificar a correção do 
software. Isto é realizado através da verificação estática e, para as partes do código que 
não puderam ser verificadas, com \rac ou \textit{Runtime Assertion Checking}. 
Ferramentas como \gospellang, \ortac e \cameleer são utilizadas para implementar essa 
técnica. O \why também é essencial para fornecer o suporte necessário a este trabalho.

Aqui, fornecemos algumas informações sobre \ocaml – a linguagem na qual nos focaremos –, 
sobre a própria verificação formal, bem como sobre \gospellang, \ortac, \cameleer e \why. Também 
apresentamos o trabalho que realizamos em um exemplo de fila (Queue), mostrando resultados 
promissores na implementação dos \monitors.

Por fim, apresentamos algumas conclusões e perspetivas para trabalhos futuros.

\keywords{
  \ocaml \and
  Verificação Formal \and
  Verificação Dinâmica \and
  \gospellang \and
  \ortac \and
  \cameleer \and
  \why
}
