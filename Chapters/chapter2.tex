%!TEX root = ../template.tex
%%%%%%%%%%%%%%%%%%%%%%%%%%%%%%%%%%%%%%%%%%%%%%%%%%%%%%%%%%%%%%%%%%%%
%% chapter2.tex
%% NOVA thesis document file
%%
%% Chapter with the template manual
%%%%%%%%%%%%%%%%%%%%%%%%%%%%%%%%%%%%%%%%%%%%%%%%%%%%%%%%%%%%%%%%%%%%

\typeout{NT FILE chapter2.tex}%

\chapter{Background}
\label{cha:background}

In this chapter, we will present the concepts and tools that are essential for the 
understanding of the work that we propose. We will start by presenting the concept of 
Formal Verification (\ref{sec:formal_verification}) to present one the pillars of software 
verification, as it is one of the fundamental concepts that we work with. 

Then, we will proceed to present \ocaml (\ref{sec:ocaml}), the language that we will be the 
focus of this dissertation. We will present its features and how it can be useful for the tools 
that we will be using. It provides an overview of the language, as well as some of the most 
relevant features, such as modules, functors and higher order functions. All of these will be 
presented with examples to better illustrate the concepts.

After that, we will present \gospellang and \cameleer (\ref{sec:gospel_and_cameleer}), two tools 
used for both static (also referenced as deductive) and dynamic (also referenced as \rac) verification. 
They are tools that are used to provide \ocaml code with formal contracts, such as preconditions, 
postconditions, invariants and other specifications. These will be explained more in depth in the 
section dedicated to them, as well as some examples for better understanding. 

The section \ref{sec:why3} will be dedicated to \why and its usages in software verification. 
In that section we also provide some insight to the solvers that are used in the platform. 

Lastly, we will present \rac (\ref{sec:rac}), a technique that is used to verify code during runtime. 
We provide details on how it works (\ref{sub:technique}), how it relates to static verification 
(\ref{sub:relation_with_static_verification}), and how it is used in \ortac (\ref{sub:ortac}), a tool 
of utmost importance for our work.

\section{Formal Verification}
\label{sec:formal_verification}

Formal Verification is the process of using mathematical methods to rigorously 
prove the correctness of a software system. Its goal is to ensure that a system 
is compliant with a given set of specifications, leaving no room for ambiguities or 
unforeseen errors. Unlike traditional testing, which can only cover a 
finite set of scenarios, formal verification provides a guarantee of 
correctness for all possible inputs and states within the specified domain~\cite{Brian_Polgreen2025}.

The use of this process requires a certain level of expertise in formal methods, such as 
logical reasoning, model checking, and theorem proving. It is a complex and time-consuming 
process, but the results are proven to be worth the effort, so much so that Formal Verification 
is used in critical systems, such as avionics~\cite{Yin_Yongfeng2010}, automotive~\cite{Rajabli2021}, 
and medical devices~\cite{Bezerra_Jonas_Santos2016}. It also helps reduce workload and costs 
of software maintenance, as it can detect and prevent errors early in the development process.

\section{OCaml}
\label{sec:ocaml}

\ocaml is a functional programming language with a strong type system and a powerful
module system. It supports imperative, functional, and object-oriented programming 
paradigms, making it a versatile choice for various types of software development. 
\ocaml's type inference system helps catch errors at compile time, reducing the 
likelihood of runtime errors and improving code reliability.

These aspects of this functional language make it a useful and resourceful tool 
for developing software that requires high standards of verification and thus, correctness.
It is language becoming more widely used for a plethora of applications.

Some of the key features of \ocaml are: its expressive type system; the pattern matching cases; 
the higher-order functions; and the module system. All these features make \ocaml an expressive 
and concise language, which can be used to develop both simple and complex systems.

\subsection{Language Overview} % (fold)
\label{sub:language_overview}

Being a functional language, \ocaml encourages the use of higher-order functions, 
which can accomodate for more generic solutions for a variety of problems. However,
\ocaml's expressive type system and powerful abstractions can lead to more robust
and maintainable code, especially in large and complex systems. The language's 
modules allow for encapsulation and organization, making code readability and organization 
easier to achieve.

As an example, consider the following code snippet, which defines a simple stack

\begin{ocamlenv}
    type 'a stack = 'a list

    let empty : 'a stack = []

    let push x s = x :: s

    let pop a =
        match a with
        | [] -> failwith "Empty stack"
        | x :: s -> x
\end{ocamlenv}

This code defines a stack data structure and its basic operations. It then can be used 
in a reusable manner throughout the codebase. As an example:

\begin{ocamlenv}
    let s = push 1 (push 2 empty)
    let t = pop s
\end{ocamlenv}

For programmers, it is a familiar and easy-to-understand way of defining and using 
data structures and algorithms, as they can be previously declared and used throughout 
the code.

\subsection{Modules and Functors} % (fold)
\label{sub:modules_and_functors}

Modules and functors are core features of \ocaml’s type and abstraction system. 
They allow for organizing and structuring code effectively, enabling encapsulation, 
reuse, and flexibility.

A module is a collection of definitions, such as types, functions, and values, 
grouped under a single name. Modules provide namespaces to prevent naming conflicts 
and enable better code organization.

\begin{ocamlenv}
    module type Stack = sig

        type 'a t

        val create : unit -> 'a t

        val push : 'a -> 'a t -> unit

        val pop : 'a t -> 'a

    end
\end{ocamlenv}

\subsection{Higher Order Functions}
\label{sub:higher_order_functions}

Higher-order functions are functions that can take other functions as arguments 
or return them as results. They are a cornerstone of functional programming in \ocaml.
They allow 
for concise and expressive code, enabling powerful abstractions and code reuse. 
Here is an example of a higher-order function in OCaml:

\begin{ocamlenv}
    let apply_twice f x = f (f x)
\end{ocamlenv}

In this example, \texttt{apply\_twice} is a higher-order function that takes a function 
\texttt{f} and an argument \texttt{x}, and applies \texttt{f} to \texttt{x} twice.

We can also take as an example the code in the previous section, adding the \texttt{filter} 
function, taking a function as an argument:

\begin{ocamlenv}
    module type Stack = sig
        ...
        val filter : ('a -> bool) -> 'a list -> 'a list
\end{ocamlenv}

Here, the \texttt{filter} function takes a predicate function and a list, and returns 
the elements of the list that satisfy the predicate. This is an example of an Higher Order 
Function, as it takes a function as an argument.

\section{GOSPEL \& Cameleer}
\label{sec:gospel_and_cameleer}

\gospellang is a specification language built with the goal of providing \ocaml code 
with formal contracts, such as preconditions, postconditions, invariants and other 
specifications. These contracts can be used to specify the intended behavior of 
specific parts of the code~\cite{Soares_Chirica_Pereira2024}, helping to ensure correctness and reliability.

\cameleer is a tool that translates the beforementioned contracts into WhyML~\cite{Pereira_Ravara2021} - 
the specification language of \why, a platform that we will talk about in 
section \ref{sec:why3}.

These tools are strongly coupled to provide a more reliable way of verifying 
\ocaml programs~\cite{Pereira2024}.

\subsection{Example}
\label{sub:example}

Let's take the \texttt{Stack} module from the previous section and add some \gospellang. 
The example should look something like this:

\begin{gospel}
    module type Stack = sig

        type 'a t
        (*@ mutable model contents: 'a list *)

        val create : unit -> 'a t
        (*@ s = create ()
            ensures s.contents = [] *)

        val push : 'a -> 'a t -> unit
        (*@ push x s
            modifies s.contents
            ensures s.contents = x :: (old s.contents) *)

        val pop : 'a t -> 'a
        (*@ r = pop s
            requires s.contents <> []
            modifies s.contents
            ensures s.contents = (List.tl (old s.contents))
            ensures r = List.hd (old s.contents) *)

    end
\end{gospel}

The \gosp{mutable}, \gosp{ensures}, \gosp{modifies}, and \gosp{requires} are all specifications, 
or contracts, defined in \gospellang. They specificy what each function should do.

The \gosp{mutable} keyword is used to define the state of the data structure, in this case, 
that the \texttt{contents} of the stack are allowed to be changed during runtime.

The keyword \gosp{ensures} is used to define the expected outcome of the function, or what the function 
should return. In the case of the \texttt{push} function, it should return the stack with 
the new element added to the top of the old values, defined by \gosp{old s.contents}.

For the \texttt{pop} function, the \gosp{requires} keyword is used to define a precondition, 
a state that must be true when the execution enters the function. Here, it is defined that 
the stack must not be empty.

Lastly, the \gosp{modifies} keyword is used to define that a certain data structure is being 
altered during the execution of the function. During \texttt{pop}, the \texttt{contents} of 
the stack are being modified, by removing the first element.

\section{Why3}
\label{sec:why3}

\ocaml in itself is a powerful language, but it lacks the tools to provide clear 
and concise specifications for the code. \why is a platform that provides a framework 
for deductive program verification. It allows for the specification of programs, such as the ones 
described previously in section~\ref{sec:gospel_and_cameleer}, to be verified using 
logical verification conditions. It works with SMT solvers, such as \zthree, \altergo, 
and \coq.

\why has its own specification language, called WhyML, which is used to define the 
contracts and the logical verification conditions. It is a language that is based on 
functional languages, as it contains constructs that are similar to those found in 
functional languages, such as \ocaml. These aspects are pattern-matching, algebraic types 
and polymorphism~\cite{Pereira}. Being a specification language means that its importance is 
in the verification of code statically, before it is exectuted.

When a program is verified using \why, it generates verification conditions that are 
sent to the solvers. These solvers then try to prove the conditions, and if they are 
proven, the program is considered correct. If not, the solvers provide counterexamples 
that can be used to debug the code. It allows for full verification of the code in terms 
of type-checking, logical verification, the use of polymorphism, and other features that 
are essencial for the proper functioning of the program.

As an example, consider the following code snippet: 

\begin{whylang}
    let rec factorial (n: int) : int
        requires { n >= 0 }
        ensures { result >= 1 }
    = if n = 0 then 1 else n * factorial (n - 1)
\end{whylang}

To verify the correctness of the \texttt{factorial} function, we can insert assertions 
to check for preconditions, postconditions, and invariants. For example: 

\begin{whylang}
    let rec factorial (n: int) : int
        assume { n >= 0 }
        = if n = 0 then 1
        else n * factorial (n - 1)
        assert { result >= 1 }
\end{whylang}

\todo[inline]{See "assume" with professor.}

This example provides a simple but really important illustration of how \rac works, as well 
as what our work will focus on. The assertions are inserted into the code, and when the 
function is executed, the assertions are checked. If they fail, an error is raised, indicating 
that the code is not behaving as expected. This is a useful technique for \rac, although 
here it is used in a static verification context, resorting to SMT solvers.

\section{Runtime Assertion Checking}
\label{sec:rac}

There are several approaches to formal verification, such as static verification, 
which is based on the analysis of static code (with an example being type checking), and dynamic 
verification, which is based on the execution of the code, with the use of 
techniques resorting to assertions and logical contracts. Runtime Assertion Checking 
(\rac) is a dynamic verification technique that consists of inserting assertions 
into the code to check for correctness during runtime.

These assertions can be used to verify preconditions, postconditions, and invariants, 
providing a way to ensure the correctness of the code during execution. For this to 
happen, the programmer (or tester), need to insert assertions to define the intended 
behaviour of functions or modules. At runtime, these assertions are checked, and if 
any of them fail, an error is raised, indicating that the code is not behaving as 
expected.

\subsection{Technique}
\label{sub:technique}

As explained before, \rac is a technique that consists of inserting assertions into 
parts of the code that are the target of verification, and then executing the program 
focusing in those same parts. Therefore, the technique consists of two parts:

\begin{itemize}
    \item Defining assertions with the intended behaviour of the code;
    \item Executing the code and verifying if the aforementioned assertions are met.
\end{itemize}

\subsubsection{Assertions}
\label{subsub:assertions}

Assertions are logical expressions that define the intended behaviour of the 
function or module they are inserted in. They consist of preconditions (\texttt{requires}), 
postconditions (\texttt{ensures}), and invariants (\texttt{invariant}).

The first are conditions that must be met before the execution enters 
the portion of code that is being verified. A use of this is to check if the input 
parameters are within the expected range, or if the data structures are in a valid 
state. Validating the input can prevent the code from ending abrutly, raising an error.

The second represent the expected results of the execution of the code, i.e. the expected 
outcome of the function or module. This can be used to check if the output is correct and 
if the data structures and modules accessed remained valid and/or in a predefined state. 

Lastly, invariants delimit the state of the variables and data structures that are 
being used in that portion of code. Here, the programmer can define the limits of 
the data structures, the expected values of the variables, and other conditions that 
must remain true. Invariants are used when \texttt{loops} (such as \texttt{for} and 
\texttt{while}) are present.

\subsubsection{Execution}
\label{subsub:execution}

When the code is executed, tools focus in verifying that the products of 
the execution are in accordance with the assertions, and if they are not, an error is 
raised, alerting the programmer or tester that the code is not behaving as expected. 
The errors raised can also be used to debug code if they provide information about the 
difference between the expected and the actual results.

\subsection{Relation with Static Verification}
\label{sub:relation_with_static_verification}

Dynamic verification has the goal of verifying code, checking for problems and errors 
that may be present. Almost as a form of testing. Static verification in not much different, 
as the goal is precisely the same. However, the approach is different. While dynamic ficuses 
on execution, static verification focuses on the code itself, analyzing it to find possible 
errors. 

Both techniques are complementary, with ones strengths being the others weaknesses.
Dynamic verification cannot run if code compilation fails - this can be enabled by static 
verification as it can catch errors like type mismatches and syntax errors. On the other hand, 
static verification on mutable pieces of code, like data structures, is much less effective 
than dynamic verification, as the states of these structures can change during execution.

\subsection{ORTAC}
\label{sub:ortac}

\ortac is a tool for \rac in \ocaml programs. It translates \ocaml module interface with 
\gospellang specifications into code to be verified~\cite{Filliatre}.

It is an essential tool for dynamic verification in \ocaml, more specifically for 
\rac. It is a bridge between the \gospellang specifications, \ocaml code, and the other tools 
used for dynamic verification. Although it uses some plugins for the testing of the code it 
generates, \ortac presents an ingenious model for the verification process.

First, the programmer writes the \ocaml code and the \gospellang specifications. Here the 
specifications are inserted in the code, allowing for a more clear and straightforward 
way of defining the intended behaviour of the code. Then, \ortac translates the code 
and the specifications into a format that can be verified by the plugins~\cite{Filliatre}. 
On their part, plugins "bombard" the code with tests, checking multiple possibilities in a 
search for errors.