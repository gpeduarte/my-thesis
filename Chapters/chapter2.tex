%!TEX root = ../template.tex
%%%%%%%%%%%%%%%%%%%%%%%%%%%%%%%%%%%%%%%%%%%%%%%%%%%%%%%%%%%%%%%%%%%%
%% chapter2.tex
%% NOVA thesis document file
%%
%% Chapter with the template manual
%%%%%%%%%%%%%%%%%%%%%%%%%%%%%%%%%%%%%%%%%%%%%%%%%%%%%%%%%%%%%%%%%%%%

\typeout{NT FILE chapter2.tex}%

\chapter{Background}
\label{cha:background}

\glsresetall

\begin{center}
  \fbox{\LARGE
    This manual is outdated and must be revised!}
\end{center}


\section{Formal Verification}
\label{sec:formal_verification}

This Chapter describes how to use the \gls{novathesis}\ template.  It is assumed that you have a working \index{installation} of \LaTeX, either local (in your own computer) or remote (in), and that you were able to generate a PDF for the default configuration of the template: a PhD thesis for \gls{FCT}.


\section{OCaml}
\label{sec:ocaml}

\subsection{Language Overview} % (fold)
\label{sub:language_overview}

Follow these steps to get started with a local \LaTeX\ installation:

\begin{enumerate}
  \item Download \LaTeX.  There are two major \LaTeX\ distributions — \href{https://miktex.org/}{\MikTeX} and \href{https://www.tug.org/texlive/}{\TeXLive} — that share lots of similarity, and \LaTeX\ documents are portable between them. This means that, for most users, both systems are equally usable.
  \begin{description}
    \item [\TeX-Live] is maintained by (La)\TeX\ developers and is certainly the best distribution you may install in your computer:  However, the default distribution will take more than 5\,GB on your hard disk… so, if you are not short on disk space, install \TeXLive!
    \item[Mik\TeX] will, by default, install only a minimal set of packages. The extra/additional packages will be installed on the fly.  Installing packages on the fly is useful if disk space is limited, but has its own caveats in the longer term.  Definitely choose \MikTeX\ if you're short on disk space.
  \end{description}
  Which one to download?  There are \href{https://tex.stackexchange.com/questions/20036/what-are-the-advantages-of-tex-live-over-miktex}{pros and cons for both distributions} so it is essentially a question of where does your heart fall first!  Mine falls to \TeXLive, but yours can fall elsewhere!  \emojiSmile
  \item Install \LaTeX. Installation of \LaTeX\ is as hard as installing any other software.  Just do your best and you will certainly succeed. 
  \item Update your \LaTeX\ installation using the \emph{\TeXLive Utility} program of the \MikTeX\ console.
  \item Download the \gls{novathesis} template by either:
  \begin{itemize}
    \item Cloning the \href{https://github.com/joaomlourenco/novathesis}{GitHub repository} with
    \begin{verbatim}    git clone --depth=1 https://github.com/joaomlourenco/novathesis.git\end{verbatim}
    or
    \item Downloading the \href{https://github.com/joaomlourenco/novathesis/archive/main.zip}{latest version from the GitHub repository as a Zip file}.
  \end{itemize}

  \item \label{it:project_available} Download additional School specific files if applicable:
  \begin{description}
    \item[Universidade do Minho (UMINHO)] download the required \emph{NewsGotT} font files from
    \url{https://github.com/joaomlourenco/novathesis-extras/raw/main/Fonts/NewsGotT.zip}\\
    then unzip the file and copy the~3~font files
\begin{flushleft}
\hspace*{0.5cm}“\verb!n015002t.ttf!”, “\verb!n015003t.ttf!”, and “\verb!n015006t.ttf!”
\end{flushleft}
    to the folder
\begin{flushleft}
\hspace*{0.5cm}“\verb!NOVAthesisFiles/FontStyles/Fonts!”.
\end{flushleft}
    \item[Escola Superior de Enfermagem do Porto (ESEP)] download the required \emph{Calibri} font files from
    \url{https://github.com/joaomlourenco/novathesis-extras/raw/main/Fonts/Calibri.zip}\\
    then unzip the file and copy the~4~font files
\begin{flushleft}
\hspace*{0.5cm}“\verb!Calibri.ttf!”, “\verb!Calibrib.ttf!”, “\verb!Calibrii.ttf!”, and “\verb!Calibriz.ttf!”
\end{flushleft}
      \noindent to the folder
\begin{flushleft}
\hspace*{0.5cm}\verb!NOVAthesisFiles/FontStyles/Fonts!.
\end{flushleft}
  \end{description}

  \item Compile the document with you favorite LaTeX processor (pdfLaTeX, XeLaTeX or LuaLaTeX).
  \begin{itemize}
    \item The main file is named “\verb!template.tex!”, but you are free to rename it as you please.
    \item Either load the main file in your favorite \href{https://en.wikipedia.org/wiki/Comparison_of_TeX_editors}{LaTeX text editor} and press the appropriate (\emph{magic}) button to generate a PDF file, or open a terminal and compile it with “\verb!latexmk -pdf template!”. If you use a \LaTeX\ text editor, please notice that the \gls{novathesis} template uses “\verb!biber!” and not “\verb!bibtex!” to process the bibliography, which means that most probably you have to open the \emph{Editor Preferences} and somewhere (depending on the Editor you are using) change “\verb!bibtex!” to “\verb!biber!”.
    \item Notice that, due to the external font sets used, \pdfLaTeX\ will not work for both \textbf{UMINHO} and \textbf{ESEP}, and you have to use either \XeLaTeX\ (“\verb!latexmk -pdfxe template!”) or \LuaLaTeX (“\verb!latexmk -pdflua template!”).
  \end{itemize}
  \item Edit the files in the “\texttt{Config}” folder.  See \autoref{tab:configuration_files}.
  \item Recompile de document.
  \item And you're done with a beautifully formatted thesis/dissertation! {\setlength{\twemojiDefaultHeight}{1.5\twemojiDefaultHeight}\emojiSmile}
\end{enumerate}

% subsection with_a_local_latex_installation (end)

\subsection{Modules and Functors} % (fold)
\label{sub:modules_and_functors}

Follow these steps to get started with a remote \LaTeX\ installation:

\begin{itemize}
  \item Download the \href{https://github.com/joaomlourenco/novathesis/archive/main.zip}{latest version from the GitHub repository as a Zip file}.
  \item Login to your favorite LaTeX cloud service. I recommend \href{https://www.overleaf.com/?r=f5160636&rm=d&rs=b}{Overleaf} but there are alternatives. These instructions apply to Overleaf and you'll have to adapt for other providers.
  \item In the menu select \fbox{New project}$\rightarrow$\fbox{Upload project}.
  \item Select “\verb!template.tex!” as the main file.
  \item Follow from Step~\ref{it:project_available} above in Section~\ref{sub:with_a_local_latex_installation} (\nameref{sub:with_a_local_latex_installation}).
\end{itemize}

% subsection with_a_remote_cloud_based_service (end)


\subsection{Higher Order Functions}
\label{sub:higher_order_functions}

\section{GOSPEL \& Cameleer}
\label{sec:gospel_and_cameleer}

The \gls{novathesis} template is organized into many files and folders. At the main level it includes the following files and folders listed in Table~\ref{tab:folders_and_files}.

\newcommand{\accessAllowed}{\includegraphics[align=c,width=1.9em]{access_allowed}}
\newcommand{\accessForbiden}{\includegraphics[align=c,width=1.9em]{dont_touch}}
\newcommand{\File}{\includegraphics[align=c,width=1.9em]{file}}
\newcommand{\Folder}{\includegraphics[align=c,width=1.9em]{folder}}


\bgroup
    \rowcolors{1}{}{GhostWhite}
      \begin{xltabular}{\textwidth}{>{\ttfamily}l>{\itshape}lcX}
        \caption{The folders and files (top level).}
        \label{tab:folders_and_files}\\
        \toprule
        \rowcolor{Gainsboro}%
        Name & Type & Access & Contents \\
        \midrule
template.tex      & \File    & \accessForbiden &
The main template file. You need to \emph{compile} this file with one of \pdfLaTeX, \XeLaTeX, or \LuaLaTeX\ to obtain the PDF file (”\texttt{template.pdf}”).  I recommend the usage of the ”\texttt{latexmk}” command or, if you use a UN*X-like OS, you may use ”\texttt{make}” (and the ggiven ”\texttt{Makefile}”).
\\
Config          & \Folder  & \accessAllowed &
Configuration files.  Please customize your template by changing the files in this folder!
\\
Chapters          & \Folder  & \accessAllowed &
Examples of document contents, including Chapters, Appendices, Annexes, Abstracts, Glossaries, Lists of Symbols, etc. Replace them with your own.
\\
Bibliography      & \Folder    & \accessAllowed &
Where all your bibliography files should be located. You may have has many bibliography files as you want.
\\
template.pdf      & \File    & \accessAllowed &
A possible result of applying \pdfLaTeX\ to the “\texttt{template.tex}” file. The look and feel of the document will depend on the parametriza\-tion/\-con\-fig\-u\-ra\-tion (e.g., School) of this template.
\\
novathesis.cls     & \File    & \accessForbiden &
The main class file.
\\
NOVAthesisFiles   & \Folder  & \accessForbiden &
Additional files for the \gls{novathesis} template.  This is where all the juice is so, unless you are a \TeX magician, don't mess up with the files and folders inside this folder.
\\
        \bottomrule
        \end{xltabular}
    % \end{longtblr}
\egroup

\bgroup
    \rowcolors{1}{}{GhostWhite}
      \begin{xltabular}{\textwidth}{>{\ttfamily}l>{\itshape}lcX}
        \caption{The configuration files (\texttt{Config} folder).}
        \label{tab:configuration_files}\\
        \toprule
        \rowcolor{Gainsboro}%
        Name & Type & Access & Contents \\
        \midrule
\texttt{0\_memoir.tex}      & \File  & \accessForbiden &
Options specific for the \texttt{memoir} class. \emph{Don't touch this file unless you know what you are doing!}
\\
\texttt{1\_novathesis.tex}  & \File  & \accessAllowed &
The main configuration file for the template, e.g., select the document type, the school, the used languages, etc.  
\\
\texttt{2\_biblatex.tex}      & \File  & \accessAllowed &
Select how your citations and bibliographic references will be printed.  The default is numbers inside square brackets, e.g. \cite{novathesis-manual}, but you can change it to other formats, such as author-year, e.g., \citeauthor{novathesis-manual}~(\citeyear{novathesis-manual}).
\\
\texttt{3\_cover.tex}		& \File & \accessAllowed & 
Configure cover contents (e.g., author's name, thesis/dissertation title, author, advisers, committee, etc)
\\
\texttt{4\_files.tex}		& \File & \accessAllowed & 
Select which files shall be included in the document as chapters, appendices, annexes, etc…
\\
\texttt{5\_packages.tex}		& \File & \accessAllowed & 
User's customization, such as loading additional packages and declare user defined commands.
\\
\texttt{6\_list\_of.tex}		& \File & \accessForbiden & 
Configure the lists to be printed (table of contents, list of figures, list of tables, list of listings, etc). \emph{Don't touch this file unless you know what you are doing!}
\\
\texttt{9\_nova\_fct.tex}	& \File & \accessAllowed & 
Configurations specific to NOVA FCT. \emph{Otherwise ignored.}
\\
\texttt{9\_ulisboa\_fmv.tex}	& \File & \accessAllowed & 
Configurations specific to ULISBOA FMV. \emph{Otherwise ignored.}
\\
\texttt{9\_ulisboa\_ist.tex}	& \File & \accessAllowed & 
Configurations specific to ULISBOA IST. \emph{Otherwise ignored.}
\\
\texttt{9\_uminho.tex}		& \File & \accessAllowed & 
Configurations specific to UMINHO (all schools). \emph{Otherwise ignored.}
\\
        \bottomrule
        \end{xltabular}
    % \end{longtblr}
\egroup


\subsection{Example}
\label(sub:example)

% section folder_structure (end)

% ===================
% = Package options =
% ===================
\section{Why3}
\label{sec:why3}

The \novathesistxt\ template can be customized by editing the files in the \texttt{Config} folder.

\newcommand{\classoption}[4]{\textbf{#1=OPT}\newline\emph{\small#2}&\textbf{#3}\newline{\small#4}\\}
\newcommand{\defaultopt}[1]{\mbox{$\Rightarrow$~\emph{Default: \texttt{#1}}}\newline}
\newcommand{\defaultit}[1][default]{($\Leftarrow$~\emph{#1})}

\section{RAC (Runtime Assertion Checking)}
\label{sec:rac}

\subsection{Technique}
\label{sub:technique}

\subsection{Relation with Static Verification}
\label{sub:relation_with_static_verification}

\section{ORTAC}
\label{sec:ortac}

%
% Please note that
% \begin{center}
%   \textbf{\large this package and template are not official for FCT/NOVA}.
% \end{center}



% \printbibliography[heading=subbibliography, segment=\therefsegment, title={\bibname\ for chapter~\thechapter}]
