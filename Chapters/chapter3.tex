%!TEX root = ../template.tex
%%%%%%%%%%%%%%%%%%%%%%%%%%%%%%%%%%%%%%%%%%%%%%%%%%%%%%%%%%%%%%%%%%%%
%% chapter3.tex
%% NOVA thesis document file
%%
%% Chapter with a short latex tutorial and examples
%%%%%%%%%%%%%%%%%%%%%%%%%%%%%%%%%%%%%%%%%%%%%%%%%%%%%%%%%%%%%%%%%%%%

\typeout{NT FILE chapter3.tex}%

\makeatletter
\newcommand{\ntifpkgloaded}{%
  \@ifpackageloaded%
}
\makeatother


\chapter{State of the Art}
\label{cha:state_of_the_art}

This chapter is reserved for the current state of software verification,
focusing on the tools that are used for this purpose. We present some 
of the tools and how they are used, as well as a brief history of the 
development of software verification, from \textit{Eiffel} to the present day.
The objective of the State of the Art is to provide a better understanding 
of the current state of the theme of this dissertation, giving practical 
examples and brief explanations of uses of both tools and techniques 
in the everyday life of software verification.

We start by presenting the current use of both static and dynamic verification tools 
(\ref{sec:combine_static_and_dynamic_verification}), and how they are used in 
combination, if that is being employed. It is an important section for us 
to understand the current position of this vision of software verification, as 
it directly impacts the work that we are proposing.

We then present the current tools and their respective uses in section
(\ref{sec:current_tools}). Here we focus in more detail on the tools that are 
being described and the languages that they emcompass. This provides a more 
"vivid" picture of the current state in the field of software verification.

Finally, we present the concept of \rac, or Runtime Assertion Checking, and how 
it is used in the context of executable specifications 
(\ref{sec:runtime_assertion_checking_executable_specifications}). It is important 
to note de difference between \rac and its derivatives of executable specifications, 
as they will be very much present in our work. We provide the reader with a brief 
example of ACSL and E-ACSL for a better understanding of the concept.

\section{Combine Static and Dynamic Verification}
\label{sec:combine_static_and_dynamic_verification}

This work that we propose in similar to that of Soares, Chirica and Pereira's~\cite{Soares_Chirica_Pereira2024}. 
This dissertation circles around \gospellang, one of the tools used in 
dynamic verification, and its relation to its static counterparts. What we propose to do with this work 
diverges from this, as we focus in combining the two in a unique technique, called \monitors, that 
verifies the program initially with static verification, and then, when that process concludes, resorts to 
dynamic verification (more specifically \rac~\cite{Soares_Chirica_Pereira2024}). It resorts not only to 
\gospellang, but also \ortac, \why and \cameleer, as well as static verification, thus being more robust and 
complete.

This cannot be done without including Formal Verification, as demonstrated by Brian and Polgreen's 
work~\cite{Brian_Polgreen2025}. It is one of the pillars of software verification, as it permits the search for correctness 
to be as precise, robust and complete as it can be.

Currently, the combination of these two techniques is either theoretical or it exists losely coupled together. 
What we propose is directed to a single and efficient way of fully verifying an \ocaml program resorting to 
tools that focus in either static or dynamic strands of software verification. Both can be used in a way that 
one strengthens the disadvantages of the other, and vice-versa.

\section{Current Tools}
\label{sec:current_tools}

In the real world, there are many tools that can be used to verify software. 
Examples such as \why, Frama-C, SPARK~\cite{Kosmatov2014} present types of tools that are used 
for verification of software. \why focuses on the verification of \ocaml programs, while Frama-C 
focuses on the verification of C programs. SPARK is a technology that focuses on the verification 
of Ada and also C/C++ programs. They provide coverage for different languages, but they all 
focus on static verification and their goal is to prove the correctness of the software.

The tools that focus on dynamic verification - more specifically, runtime assertion checking, 
or \rac for short - are also used for the same purpose, with a different approach~\cite{Maurica2018}.
\begin{itemize}
  \item JML, or Java Modeling Language, is a language that is used to express specifications of Java~\cite{Leavens2008}.
  \item ACSL, or ANSI/ISO C Specification Language, is a language used to express specifications of C~\cite{Baudin}.
  \item SPARK focuses on the verification of Ada programs~\cite{Spark2022}.
\end{itemize}
These are some of the examples of tools that are used for dynamic verification. Their objective 
is to verify software during runtime, checking for behaviour that is not expected.

\textit{Eiffel} is a programming language that has a built-in assertion language~\cite{Maurica2018}. 
At the time of its creation, it was one of the first languages to have this feature, and it presented 
a new way of thinking about software verification. Designed in the 1980s, it was a language that 
focused on the development of software that was correct by construction. This means that the 
software was developed in a way that it was correct from the beginning. It had a built-in assertion 
language that allowed the programmer to express the specifications of the software. Contracts, 
static type checking, garbage collection are just some of the features that \textit{Eiffel} has. 

Due to the fact that this language focused in the development of software that was correct from the 
start, it influenced a lot of the tools that are used today. The tools above mentioned are 
just some examples of tools that were influenced by \textit{Eiffel}.

\section{Runtime Assertion Checking - Executable Specifications}
\label{sec:runtime_assertion_checking_executable_specifications}

Although dynamic verification is in itself a powerful technique for the verification of systems, 
developers found that it was necessary to have a way to express specifications directly in the 
code. This is where executable portions of specification languages come in. 

ACSL, as mentioned before, expresses specifications, but it is not executable. Thus, 
E-ACSL was created. It is a tool that is used to express specifications 
amidst the code, making those specifications executable. This tool translates C code with 
specifications into C code that is instrumented with assertions. It is a conversion tool 
that makes verification simpler and more efficient~\cite{Maurica2018}. This is one of the 
principles that we will use in our work, as E-ACSL is only designed for C programs.

This is what we hope to accomplish with our work. We want to create a tool that is able to 
express specifications into \ocaml programs, making them executable. We hope to produce a 
\gospellang executable specification language that is able to express the specifications 
produced by the programmer, and converting them into assertions, inserted in the right 
places of the code. Thus, E-GOSPEL will provide a step forward in the verification of 
\ocaml, allowing programmers specialized in this language to use the tool to verify their 
software in a more efficient and robust way. It is also a step towards \monitor implementation, 
as it is a technique that has some promising results in an area where \monitors are 
inserted in.

Executable Specification Languages bridge the gap between what the programmer wants the 
program to do and what the system will try to verify in the code. It is a way to express 
the desire of the programmer directly in the code, making it easier to verify the software as 
verification and code are being executed in the same environment.