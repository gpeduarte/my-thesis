%!TEX root = ../template.tex
%%%%%%%%%%%%%%%%%%%%%%%%%%%%%%%%%%%%%%%%%%%%%%%%%%%%%%%%%%%%%%%%%%%%
%% chapter3.tex
%% NOVA thesis document file
%%
%% Chapter with a short latex tutorial and examples
%%%%%%%%%%%%%%%%%%%%%%%%%%%%%%%%%%%%%%%%%%%%%%%%%%%%%%%%%%%%%%%%%%%%

\typeout{NT FILE chapter3.tex}%

\makeatletter
\newcommand{\ntifpkgloaded}{%
  \@ifpackageloaded%
}
\makeatother


\chapter{State of the Art}
\label{cha:state_of_the_art}

\todo[inline]{What to write in this section?}

\section{Combine Static and Dynamic Verification}
\label{sec:combine_static_and_dynamic_verification}

This work that we propose in similar to that of Soares, Chirica and Pereira's~\cite{Soares_Chirica_Pereira2024}. 
  This work circle around \gospellang, one of the tools used in 
dynamic verification, and its relation to its static counterparts. What we propose to do with this work 
diverges from this, as we focus in combining the two in a unique technique, called \monitors, that 
verifies the program initially with static verification, and then, when that process concludes, resorts to 
dynamic verification (more specifically \rac~\cite{Soares_Chirica_Pereira2024}). It resorts not only to 
\gospellang, but also \ortac, \why and \cameleer, as well as static verification, thus being more robust and 
complete.

This cannot be done without including Formal Verification, as demonstrated by Brian and Polgreen's 
work~\cite{Brian_Polgreen2025}. It is one of the pillars of software verification, as it permits the search for correctness 
to be as precise, robust and complete as it can be.

Currently, the combination of these two techniques is either theoretical or it exists losely coupled together. 
What we propose is directed to a single and efficient way of fully verifying an \ocaml program resorting to 
tools that focus in either static or dynamic strands of software verification. Both can be used in a way that 
one strengthens the disadvantages of the other, and vice-versa.