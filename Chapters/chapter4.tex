%!TEX root = ../template.tex
%%%%%%%%%%%%%%%%%%%%%%%%%%%%%%%%%%%%%%%%%%%%%%%%%%%%%%%%%%%%%%%%%%%%
%% chapter4.tex
%% NOVA thesis document file
%%
%% Chapter with lots of dummy text
%%%%%%%%%%%%%%%%%%%%%%%%%%%%%%%%%%%%%%%%%%%%%%%%%%%%%%%%%%%%%%%%%%%%

\typeout{NT FILE chapter4.tex}%

\chapter{Preliminary Results}
\label{cha:preliminary_results}

For our research and introduction on the topic of static and dynamic verification, 
we had to prepare and test an example that would be used to demonstrate the 
execution model of these techniques of verification. 

The chosen example was a \texttt{Queue} implementation in \ocaml, 
verified in \why and \gospellang. The \texttt{Queue} implementation 
consists of two lists, one for the front of the queue and another for 
the back. Its functioning is quite simple: it has two operations, 
\texttt{push} and \texttt{pop}, that add and remove elements from the \texttt{Queue}. 
This approach uses the first \text{List} as the elements that are 
being removed through the \texttt{pop} operation, and the second \text{List} 
as the elements that are being added through the \texttt{push} operation.

\section{Queue Example}
\label{sec:queue_example}

The implementation of a \gosp{Queue} we opted for a double list approach, 
where the first list is the front of the queue and the second list is the back.
The principle behind this implementation is described above, where the 
\gosp{push} operation adds elements to the back of the queue and the 
\gosp{pop} operation removes elements from the front of the queue.

\subsection{Implementation}
\label{subsec:implementation}

Below is our implementation of the \gosp{Queue} in \ocaml:

\begin{gospel}
    open List

    type 'a queue = {
        mutable front : 'a list;
        mutable back : 'a list;
        mutable size : int;
    }

    let[@logic] is_empty q = q.size = 0

    let make () =
    { front = []; back = []; size = 0 }

    let pop a = 
    let x =
        | [] -> raise Not_found
        | [ x ] ->
            a.front <- List.rev a.back;
            a.back <- [];
            x
        | x :: xs ->
            a.front <- xs;
            x
    in
    a.size <- a.size - 1;
    x

    let push a x =
        if is_empty a then a.front <- [ x ] else a.back <- x :: a.back;
        a.size <- a.size + 1
\end{gospel}

As one can observe, we opted for a definition of \gosp{Queue} as a record type, 
with three fields: \gosp{front}, \gosp{back} and \gosp{size}. The \gosp{front} 
and \gosp{back} fields are lists of elements of type \gosp{'a} - meaning a certain type 
not yet defined -, and the \gosp{size} field is an integer that represents the 
number of elements in the \gosp{Queue}.

The \gosp{is_empty} function is a logical function that returns \gosp{true} if the 
size field of the \gosp{Queue q} is equal to zero, and \gosp{false} otherwise. 

The \gosp{make} function is a constructor for the \gosp{Queue} type, that returns 
a new \gosp{Queue} with the \gosp{front} and \gosp{back} fields as empty lists.

The \gosp{pop} function removes the first element of the \gosp{Queue} and returns it. 
If the \gosp{Queue} is empty, it raises a \gosp{Not_found} exception. If the \gosp{front} 
only has one element, it reverses the \gosp{back} list and assigns it to the \gosp{front}, 
clearing the \gosp{back} list in the process and returning the first value. If the 
\gosp{front} has more than one element, it removes the first element of the \gosp{front} 
list and returns it. Finally, it decrements the \gosp{size} field by one.

The \gosp{push} function adds an element to the \gosp{Queue}. If the \gosp{Queue} is empty, 
it assigns the element to the \gosp{front} list. Otherwise, it adds the element to the 
\gosp{back} list. In both cases, it increments the \gosp{size} field by one.

\subsection{Specification}
\label{subsec:specification}

With the implementation of the \gosp{Queue}, we can now specify the intended behavior 
of its operations. We used \gospellang contracts to introduce the necessary specifications 
for further testing. Below is our specification:

\begin{gospel}
    type 'a t
    (*@ mutable model view: 'a list *)

    val is_empty : 'a t -> bool
    (*@ b = is_empty a 
        ensures b <-> t.view = []*)

    val make : int -> 'a -> 'a t
    (*@ t = make ()
        ensures t.view = []*)

    val pop : 'a t -> 'a
    (*@ a = pop t 
        modifies t.view
        requires t.view <> []*
        ensures t.view = 
            if old t.view = []
            then [] 
            else List.tl (old t.view)
        ensures if old t.view = [] then false
            else a = List.hd (old t.view)*)

    val push : 'a t -> 'a -> 'a
    (*@ push t a
        modifies t.view
        ensures t.view = append_last a (old t.view)*)
\end{gospel}

This interface has all the specifications that present the behavior of our \gosp{Queue} 
implementation. We found it necessary to add a \gosp{model view} field to the \gosp{Queue} 
type, to allow for the visualization of its elements as a simple \ocaml list. 

The \gosp{is_empty} function returns \gosp{true} if the \gosp{Queue} is empty, and 
\gosp{false} otherwise. The \gosp{<->} operator is a logical equivalence operator, 
meaning that the value of \gosp{b} is equivalent to the expression of \gosp{t.view = []}. 

The \gosp{make} function creates a new \gosp{Queue} with an empty \gosp{view} field. 
It ensures that the creation of a new \gosp{Queue} results in an empty one, as 
defined by \gosp{ensures} clause.

The \gosp{pop} function removes the first element of the \gosp{Queue} and returns it. 
This is the function with more complex specifications. It modifies the \gosp{view} field 
with the \gosp{modifies} clause, and requires that the \gosp{view} field is not empty, with 
\gosp{requires}. For the postconditions, the specifications have to ensure that, first, 
the \gosp{t.view} field remains the same if there are no elements in the \gosp{Queue} 
(\gosp{ensures t.view = if old t.view = [] then [] else List.tl (old t.view)}), and, secondly, 
that the returned element is the first element of the \gosp{Queue} (\gosp{ensures if old t.view 
= [] then false else a = List.hd (old t.view)}).

Finally, \gosp{push} adds an element to the \gosp{Queue}. It modifies the \gosp{view} field 
and ensures that the \gosp{t.view} field is the result of appending the new element to the 
last position of the \gosp{Queue}, resorting to the \gosp{append_last} function.