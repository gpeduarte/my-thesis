%!TEX root = ../template.tex
%%%%%%%%%%%%%%%%%%%%%%%%%%%%%%%%%%%%%%%%%%%%%%%%%%%%%%%%%%%%%%%%%%%%
%% chapter4.tex
%% NOVA thesis document file
%%
%% Chapter with lots of dummy text
%%%%%%%%%%%%%%%%%%%%%%%%%%%%%%%%%%%%%%%%%%%%%%%%%%%%%%%%%%%%%%%%%%%%

\typeout{NT FILE chapter4.tex}%

\chapter{Preliminary Results}
\label{cha:preliminary_results}

For our research and introduction on the topic of static and dynamic verification, 
we had to prepare and test an example that would be used to demonstrate the 
execution model of these techniques of verification. 

The chosen example was a \texttt{Queue} implementation in \ocaml, 
verified in \why and \gospellang. The \texttt{Queue} implementation 
consists of two lists, one for the front of the queue and another for 
the back. Its functioning is quite simple: it has two operations, 
\texttt{push} and \texttt{pop}, that add and remove elements from the \texttt{Queue}. 
This approach uses the first \text{List} as the elements that are 
being removed through the \texttt{pop} operation, and the second \text{List} 
as the elements that are being added through the \texttt{push} operation.

\section{Queue Example}
\label{sec:queue_example}

\todo[inline]{Present the queue example: finish the implementation and show the results.}

\begin{itemize}
    \item DV
    \item RAC
\end{itemize}