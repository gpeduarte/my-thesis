%!TEX root = ../template.tex
%%%%%%%%%%%%%%%%%%%%%%%%%%%%%%%%%%%%%%%%%%%%%%%%%%%%%%%%%%%%%%%%%%%
%% chapter1.tex
%% NOVA thesis document file
%%
%% Chapter with introduction
%%%%%%%%%%%%%%%%%%%%%%%%%%%%%%%%%%%%%%%%%%%%%%%%%%%%%%%%%%%%%%%%%%%

\typeout{NT FILE chapter1.tex}%

\chapter{Introduction}
\label{cha:introduction}

\section{Motivation/Context}
\label{sec:motivation_context}

Errors are closely connected with human performance, and they are a part of everyday life. 
They can be found in all areas of human activity, and they can have different consequences, 
from minor inconveniences to catastrophic events. In the context of software development, 
errors can have a significant impact on the quality of the software, leading to financial losses, 
loss of reputation, and even loss of life. The complexity of software systems has been increasing 
over the years, and it is becoming more and more challenging to develop error-free software. 
This complexity is due to the increasing size of software systems, the increasing number of features, 
and the increasing number of interactions between different components. As a result, software 
developers are facing new challenges in developing software that is reliable, secure, and efficient.

Software verification consists of checking whether a software system meets its requirements and
specifications. In the particular case of \ocaml programs, static and dynamic verification are two powerful 
techniques that can be used to verify the correctness of \ocaml programs. Static verification consists of 
analyzing the program's code without executing it, using techniques such as type checking, abstract 
interpretation, and model checking. These techniques can detect potential errors and prove properties 
about the program, such as the absence of certain types of runtime errors.

Dynamic verification, on the other hand, involves executing the program and observing its behavior 
to ensure it meets its specifications. Techniques such as testing, runtime assertion checking, and 
formal methods like model-based testing can be used to dynamically verify \ocaml programs. 
These techniques can help identify errors that may not be detectable through static analysis alone.

The two can  be combined to provide a more exhaustive verification of \ocaml programs, 
collaborating for a more structured and correct code. This can even be more useful in the context of 
systems that have a direct impact in everyday life.

For example, \monitors follow a set of steps to ensure that the system is correctly implemented. 
First, static verification is performed to ensure that the code may not have any errors during compilation.
Then, for the parts of the code that the previous step could not verify, dynamic verification is performed 
to verify the code's behavior during execution. This process is repeated until the system is fully verified.

\section{Problem Definition}
\label{sec:problem_definition}

This work has the goal of assert the stability and validity of combining static and dynamic verification techniques 
to verify \ocaml programs extensively, to the point whether the system is fully verified, both statically and dynamically.
This goal may arise questions such as:

\begin{quote}
    \emph{Is it possible to combine static and dynamic verification for \ocaml programs?}
\end{quote}

The question above introduces a large scope, bigger than what can be covered in this work. 
As such, we can split it into smaller questions:

\begin{enumerate}
    \item Identify an executable subset of \gospellang;
    \item Should we use \rac when Deductive Verification (Static Verification) is not enough?
\end{enumerate}

Work in this dissertation will focus on the second question, as it emcompasses both static 
and dynamic verification techniques in a single process/technique. It is important to note that 
the described process in the second question will appear along the work, as it basically describes 
the process of \monitors, one of the main focus here.

For a more visual example, consider the following code snippet:

\begin{gospel}
    while B do S done
    (* @invariant I *)
\end{gospel}

We can note that the invariant $I$ can be defined as a \monitor $M[[I]]$ - it will be verified statically, 
and if it is not possible, a dynamic verification will be performed.

Our work will focus on turning the above code into a more practical format, as we now present:

\begin{gospel}
    while B do
        assert M[[I]];
        S
        assert M[[I]];
    done
\end{gospel}

This change will allow us to verify the invariant in a more practical way by translating contracts 
into built-in code assertions. This will produce both code and its verification process in the 
same file and in a way that it can be verified semi-automatically. 

For this to be done, we will have to understand tools like \cameleer and \ortac and their capacity 
to generate \monitors for dynamic verification when static verification fails.

\subsection{Static Verification}
\label{sub:static_verification}

Source code can be verified without the machine running it. This process allows to programmer 
to find error in type-checking, abstract interpretation, model checking, theorem proving and symbolic execution.
This is a form of static verification - verifying the correctness of code without the need to run it.
For programmers and testers, it is helpful to verify if the code being written/analysed is correct without 
wasting resources (time, CPU, RAM) by running the program, and static verification techniques provide just the solution.

Static verification can enhance early error detection during the development process, when the code is still 
being written, aided by the exhaustive analysis of the code, that covers all logical paths and scenarios for 
a given domain. It is a resource-efficient approach that gives the garantee of correctness of critical 
pieces of code. This leads to an overall efficiency in verifying code.

\subsection{Runtime Assertion Checking}
\label{sub:rac}

Runtime Assertion Checking (\rac) is a powerful technique that can be used to verify preconditions, postconditions, and 
invariants, proving to be of utmost importance in the verification of programs, specially 
in the context of the \ocaml language~\cite{Filliatre}, where the type system, 
although powerful, may not be enough to ensure the correctness of the code.

\rac is used during runtime, which by itself means that there needs to be a precondition required 
by the programmer: the code must be executable. In other words, \rac cannot be used if the code 
has errors that prevent it from being executed. This technique is better suited for verifying dynamic 
structures and algorithms, as these are mutable and may change during execution. This is a limitation 
of static verification, which is unable to verify mutable pieces of code.

Although very powerful, \rac is not enough to ensure the correctness of the code. One of its 
limitations is errors that can arise from the code that is static. For example, it may be unable 
to type-check code, check declarations or definitions, or verifying that the code is not going to 
compile.

This is where static verification comes in. It is a technique that can be used to verify the 
correctness of the code before it is executed. As explained in \ref{sub:static_verification}, this is done by analyzing the code without 
executing it, using techniques such as type checking, abstract interpretation, and model checking.
This technique can detect potential errors during compilation, thus ensuring that the code is 
able to run. Both dynamic and static verification can be combined to provide a more exhaustive 
and complete verification of the code.

One example of cooperation between the two techniques is applied in \monitors - a technique that, 
in first instance, uses static verification to verify some code, and then uses dynamic verification 
to verify the parts of the code that previously were unable to be verified. This process is repeated until 
the code is fully verified.

\section{Expected Contributions}
\label{sec:expected_Contributions}

This work aims to contribute to the field of software verification of \ocaml programs. 
We will focus first in developing a set of cases that will be abled to be verified both 
statically and dynamically. The second step would be to implement \monitors that can help 
in the verification process. Thus, the main tasks are:

\begin{itemize}
    \item Research and identify an executable set of \gospellang - E-GOSPEL, as in Executable \gospellang;
    \item Implement \monitors that can help in the verification process;
    \item Evaluate the effectiveness of the verification process with the aforementioned 
    sets.
\end{itemize}

\section{Report Structure}
\label{sec:report_structure}

\begin{itemize}
    \item In Chapter \ref{cha:background} we present the necessary information to understand
        the context of the work, such as the tools, techniques and concepts involved. All 
        the information presented in this chapter is essential to understand the work.
    \item Chapter \ref{cha:state_of_the_art} presents the existing work in the area of 
        both static and dynamic verification of \ocaml programs and how they are currently 
        combined to verify software systems.
    \item Chapter \ref{cha:preliminary_results} will focus on presenting preliminary 
        results of some exercises that were performed. These exercises consist of verifying 
        the effectiveness of verification in a \texttt{Queue} implementation in \ocaml. 
        Resorting to \gospellang, \cameleer and \ortac, this is an essential chapter to understand 
        what can be done to combine both static and dynamic verification.
    \item Chapter \ref{cha:work_plan} will present the scheduled plan of work ahead and explain 
    birefly how we will manage it.
\end{itemize}