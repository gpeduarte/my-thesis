%!TEX root = ../template.tex
%%%%%%%%%%%%%%%%%%%%%%%%%%%%%%%%%%%%%%%%%%%%%%%%%%%%%%%%%%%%%%%%%%%
%% chapter1.tex
%% NOVA thesis document file
%%
%% Chapter with introduction
%%%%%%%%%%%%%%%%%%%%%%%%%%%%%%%%%%%%%%%%%%%%%%%%%%%%%%%%%%%%%%%%%%%

\typeout{NT FILE chapter1.tex}%

\chapter{Introduction}
\label{cha:introduction}

\section{Motivation/Context}
\label{sec:motivation_context}

Errors are closely connected with human performance, and they are a part of everyday life. 
They can be found in all areas of human activity, and they can have different consequences, 
from minor inconveniences to catastrophic events. In the context of software development, 
errors can have a significant impact on the quality of the software, leading to financial losses, 
loss of reputation, and even loss of life. The complexity of software systems has been increasing 
over the years, and it is becoming more and more challenging to develop error-free software. 
This complexity is due to the increasing size of software systems, the increasing number of features, 
and the increasing number of interactions between different components. As a result, software 
developers are facing new challenges in developing software that is reliable, secure, and efficient.

Software verification consists of checking whether a software system meets its requirements and
specifications. In the particular case of \ocaml programs, static and dynamic verification are two powerful 
techniques that can be used to verify the correctness of \ocaml programs. Static verification consists of 
analyzing the program's code without executing it, using techniques such as type checking, abstract 
interpretation, and model checking. These techniques can detect potential errors and prove properties 
about the program, such as the absence of certain types of runtime errors.

Dynamic verification, on the other hand, involves executing the program and observing its behavior 
to ensure it meets its specifications. Techniques such as testing, runtime assertion checking, and 
formal methods like model-based testing can be used to dynamically verify \ocaml programs. 
These techniques can help identify errors that may not be detectable through static analysis alone.

The two can  be combined to provide a more exhaustive verification of \ocaml programs, 
collaborating for a more structured and correct code. This can even be more useful in the context of 
systems that have a direct impact in everyday life.

For example, \textit{monitors} follow a set of steps to ensure that the system is correctly implemented. 
First, static verification is performed to ensure that the code may not have any errors during compilation.
Then, for the parts of the code that the previous step could not verify, dynamic verification is performed 
to verify the code's behavior during execution. This process is repeated until the system is fully verified.

\section{Problem Definition}
\label{sec:problem_definition}

This work has the goal of assert the stability and validity of combining static and dynamic verification techniques 
to verify \ocaml programs extensively, to the point whether the system is fully verified, both statically and dynamically.
This goal may arise questions such as:

\begin{quote}
    \emph{Is it possible to combine static and dynamic verification for \ocaml programs?}
\end{quote}

*insert more questions here*

\todo[inline]{See with professor what other questions may arise and how to present them}

\begin{itemize}
    \item State the usefulness of the thesis theme to solving real world problems
    \item Present questions that may arise
\end{itemize}

\subsection{RAC}
\label{sub:rac}

Runtime Assertion Checking (\rac) is a dynamic verification technique that consists of 
inserting assertions - logical expressions that must be true at a certain point in the 
execution - into the code to check for correctness during runtime. If these assertions fail, 
an error is raised, indicating that the code is not behaving as expected.

This is a powerful technique that can be used to verify preconditions, postconditions, and 
invariants, proving to be of utmost importance in the verification of programs, especially 
in the context of the \ocaml language, where the type system, although powerful, may not be 
enough to ensure the correctness of the code.

\rac is used during runtime, which by itself means that there needs to be a precondition required 
by the programmer: the code must be executable. In other words, \rac cannot be used if the code 
has errors that prevent it from being executed.

This technique is better suited for verifying dynamic structures and algorithms, as these are 
mutable and may change during execution. This is a limitation of static verification, which 
is unable to verify mutable pieces of code.

Although very powerful, \rac is not enough to ensure the correctness of the code. One of its 
limitations is errors that can arise from the code that is static. For example, it may be unable 
to type-check code, check declarations or definitions, or verifying that the code is not going to 
compile.

This is where static verification comes in. It is a technique that can be used to verify the 
correctness of the code before it is executed. This is done by analyzing the code without 
executing it, using techniques such as type checking, abstract interpretation, and model checking.
This technique can detect potential errors during compilation, thus ensuring that the code is 
able to run. Both dynamic and static verification can be combined to provide a more exhaustive 
and complete verification of the code.

One example of cooperation between the two techniques is \texttt{Monitors} - a technique that, 
in first instance, uses static verification to verify some code, and then uses dynamic verification 
to verify the parts of the code that previously were not verified. This process is repeated until 
the code is fully verified.

\section{Expected Contributions}
\label{sec:expected_Contributions}

\begin{itemize}
    \item Describe how this work is expected to help with the current problems
    and how it can be a resourceful asset
\end{itemize}

\section{Report Structure}
\label{sec:report_structure}

\begin{itemize}
    \item In Chapter \ref{cha:background} we present the necessary information to understand
        the context of the work, such as the tools, techniques and concepts involved. All 
        the information presented in this chapter is essential to understand the work.
    \item Chapter \ref{cha:state_of_the_art} presents the existing work in the area of 
        both static and dynamic verification of \ocaml programs and how they are currently 
        combined to verify software systems.
    \item Chapter \ref{cha:preliminary_results} will focus on presenting preliminary 
        results of some exercises that were performed. These exercises consist of verifying 
        the effectiveness of verification in a \texttt{Queue} implementation in \ocaml. 
        Resorting to \gospellang, \cameleer and \ortac, this is an essential chapter to understand 
        what can be done to combine both static and dynamic verification.
    \item Chapter \ref{cha:work_plan} will present the scheduled plan of work ahead and explain 
    birefly how we will manage it.
\end{itemize}